

\documentclass[a4paper,12pt,spanish]{book}


\usepackage{./estilos/estiloBase} % Basicamente son todas las
                                  % librerias usadas. En caso de que
                                  % falten librerias se van añadiendo
                                  % al fichero.
\usepackage{./estilos/colores}  % Algunos colores ya generados, para
                                % los algunos estilos más avanzados.
\usepackage{./estilos/comandos} % Algunos comandos personalizados

\graphicspath{{./imagenes/}} % Indicamos la ruta donde se encuentran
                             % las imagenes, para ahorrarnos la ruta
                             % completa, y solo modificar aquí si en
                             % un momento dado lo movemos
\usepackage{enumitem}
\bibliographystyle{ieeetr}

\setlength\parindent{30pt}

\begin{document}

% Renombramos las figuras y las tablas
\renewcommand{\figurename}{Figura}
\renewcommand{\listfigurename}{Indice de figuras}
\renewcommand{\tablename}{Tabla}
\renewcommand{\listtablename}{Indice de tablas}

\pagestyle{empty}
% -*-portada.tex-*-
% Este fichero es parte de la plantilla LaTeX para
% la realización de Proyectos Final de Carrera, protejido
% bajo los términos de la licencia GFDL.
% Para más información, la licencia completa viene incluida en el
% fichero fdl-1.3.tex

% Fuente tomada del PFC 'libgann' de Javier Vázquez Púa

\begin{titlepage}

  \begin{center}

    
    
    \vspace{2.0cm}
    
    \LARGE{\textbf{ESCUELA SUPERIOR DE INGENIERIA}} 
    
    \vspace{1.0cm}
    
    \Large{\textbf{INGENIERO  EN INFORMATICA  } }\\
    
    \vspace{3.0cm}
    
    \Large{TITLO} \\
    
    \vspace{2.0cm}
    
    \Large{Rafael Morales GonzAlez} \\
  
    \vspace{0.5cm}

    \large{\today}
    
  \end{center}
  
\end{titlepage}

\cleardoublepage

% -*-primerahoja.tex-*-
% Este fichero es parte de la plantilla LaTeX para
% la realización de Proyectos Final de Carrera, protejido
% bajo los términos de la licencia GFDL.
% Para más información, la licencia completa viene incluida en el
% fichero fdl-1.3.tex

% Fuente tomada del PFC 'libgann' de Javier Vázquez Púa

\begin{center}

 

  \vspace{2.0cm}

  \Large{ESCUELA SUPERIOR DE INGENIERIA} \\

  \vspace{1.0cm}

  \large{INGENIERO EN INFORMATICA } \\

  \vspace{1.0cm}

  \large{TITULO} \\

  \vspace{1.0cm}

\end{center}

\begin{itemize}
\item \large{Departamento lenguajes y ciencias de la computaci\'on}
\item \large{Director del proyecto: z}
\item \large{Codirector del proyecto: }
\item \large{Autor del proyecto: Rafael Morales Gonz\'alez}
\end{itemize}

\vspace{1.0cm}

\begin{flushright}
  \large{M\'alaga, \today} \\

  \vspace{2.5cm}

  \large{Fdo: Rafael Morales Gonz\'alez}
\end{flushright}

\cleardoublepage
\pagestyle{plain}

\frontmatter % Introducción, índices ...

% -*-previo.tex-*-
% Este fichero es parte de la plantilla LaTeX para
% la realización de Proyectos Final de Carrera, protejido
% bajo los términos de la licencia GFDL.
% Para más información, la licencia completa viene incluida en el
% fichero fdl-1.3.tex

% Copyright (C) 2009 Pablo Recio Quijano 

\section*{Agradecimientos}

Me gustaria agradecer y/o dedicar este texto a ...

\cleardoublepage


\cleardoublepage

\tableofcontents
\listoffigures
\listoftables

\mainmatter % Contenido en si ...
%\pagestyle{headings}
\nocite{*}
\chapter{Introduccion}
CAP 1




\chapter{Estado del arte}
CAP 2

\chapter{Desarrollo del proyecto}
CAP 3

\chapter{Aplicacion}
CAP 4

\chapter{Conclusiones y futuras mejoras}
CAP 5


\backmatter % Apéndices, bibliografia ...


\clearpage


\addcontentsline{toc}{chapter}{Bibliografía y referencias}

\bibliography{bibliografia}
%\begin{thebibliography}{9}
 	\bibitem{azuma} A survey of Augmented Reality�, Ronald T. Azuma, 1997.
	\bibitem{Taxonomy}P. Milgram and F. Kishino. A Taxonomy of Mixed Reality Visual Displays. IEICE Trans. Infor-
mation Systems, vol. E77-D, no. 12, pp. 1321-1329, 1994.
	\bibitem{Bimber} O. Bimber, R. Rakar, �Spatial Augmented Reality. Merging Real and Virtual Worlds�, 
2005.
	\bibitem{ARToolKit}ARToolKit. ARToolKit Main Site. http://www.hitl.washington.edu/artoolkit/.
	\bibitem{FLARToolKit}FLARToolKit. FLARToolKit Main Site.http://www.libspark.org/wiki/saqoosha/FLARToolKit/en
	\bibitem{artoolkitplu} \url{http://studierstube.icg.tugraz.at/handheld_ar/artoolkitplus.php}
	\bibitem{Cawood}S. Cawood, M. Fiala, �Augmented Reality: A practical guide�, 2008.
	\bibitem {RAmundo}\url { http://library.abb.com/global/scot/scot271.nsf/veritydisplay/3526803755bb1020c125728f004c902f/$File/70-72%201M718_SPA72dpi.pdf}
	\bibitem {WikiRA}\url {http://en.wikipedia.org/wiki/Augmented_reality}
	\bibitem {WikiIP}\url {http://en.wikipedia.org/wiki/Thresholding_(image_processing)}
	\bibitem {Augmentes}\url {http://ael.gatech.edu/lab/}
	\bibitem{camara} \url {http://upcommons.upc.edu/pfc/bitstream/2099.1/15354/1/73353.pdf}
	\bibitem {esa} \url {http://www.esa.int/esaCP/SEMHSTSXXXG_index_0.html}
	\bibitem{Villaplana}C�sar Carri�n Villaplana. Proyecto final de carrera: �Desarrollo de un sistema de 
Realidad Aumentada para el tratamiento de fobias a  animales peque�os�. Universidad 
Polit�cnica de Valencia. Director: M. Carmen Juan Lizandra. 2004 
	\bibitem{Caudell}T. P. Caudell, and D. W. Mizell, �Augmented Reality: An Application of Heads-Up 
Display Technology to Manual Manufacturing Processes�, Proceedings of 1992 IEEE Hawaii 
International Conference on Systems Sciences, 1992, pp 659-669. 
	\bibitem{vuzix}\url{http://www.vuzix.com/}
	\bibitem{vc}\url{http://msdn.microsoft.com/en-us/library/60k1461a.aspx}
	\bibitem{shorter} \url{http://goo.gl/}
	\bibitem{RAyRV} Realidad Aumentada y Realidad Virtual \url{http://www.slideshare.net/axland/realidad-aumentada-y-realidad-virtual}
	\bibitem{Portal}Portal de la Realidad Aumentada.\url{ http://www.augmented-reality.org/}
	\bibitem{Mullen}Mullen, Tonny . Prototyping Augmented Reality. Sybex. Oct. 2011
	\bibitem{militares}\url{http://gizmologia.com/2010/06/la-realidad-aumentada-para-militares-podria-cambiar-la-forma-en-la-que-se-combate-en-la-actualidad}
	\bibitem{DARPA} DARPA \url{http://www.fayerwayer.com/2011/04/darpa-encarga-confeccion-de-gafas-de-realidad-aumentada-para-militares/}
	\bibitem{estereo}\url{http://es.wikipedia.org/wiki/Estereoscop\%C3\%ADa}
	\bibitem{RAjeuazarru}\url{http://www.jeuazarru.com/docs/Realidad_Aumentada.pdf}
	\bibitem{Cawood} S. Cawood, M. Fiala, �Augmented Reality: A practical guide�, 2008
	\bibitem{Somolinos} J. A. Somolinos S�nchez, �Avances en rob�tica y visi�n por computador�, 2002
	\bibitem{Tateno} K. Tateno, I. Kitahara, Y. Ohta, �A  Nested Marker for Augmented Reality�, 2007
	\bibitem{Javidi}B. Javidi, �Image Recognition and Classification.  Algorithms, Systems and 
Applications�, 2002.
	\bibitem{mars} MARS \url{ http://graphics.cs.columbia.edu/projects/mars/mars.html}
	\bibitem{problemas} \url{http://sabia.tic.udc.es/gc/Contenidos%20adicionales/trabajos/3D/RealidadAumentada/1.3.problemas.html}
	\bibitem{arpa} \url{http://www.arpa-solutions.net/}
	\bibitem{Tinmith-Metro}Tinmith-Metro  \url{http://www.tinmith.net/} 
	\bibitem{arquake} ARQuake \url{http://wearables.unisa.edu.au/arquake/ }
	\bibitem{archeoguide} Archeoguide \url{ http://archeoguide.intranet.gr/ }
	\bibitem{layar} Layar \url{ http://www.layar.com/}
	\bibitem{aplicaciones} \url{http://www.adarveproducciones.com/uploads/ficha/fichero/APLICACIONES\%20DE\%20LA\%20REALIDAD\%20AUMENTADA_126.pdf}
	\bibitem{cisgalicia} \url{http://www.cisgalicia.org/pages/idc/descargas/perfil_idc.pdf}
	\bibitem{com} \url{http://www.wikilearning.com/tutorial/guia_de_directx/5096}
	\bibitem{windows} \url{http://msdn.microsoft.com/es-es/library/60k1461a(v=vs.90).aspx}
	\bibitem{directx1} \url{http://www.directxtutorial.com/}
	\bibitem{directx2} \url{http://en.wikipedia.org/wiki/DirectX}
	\bibitem{OpenCVB} Gary R. Bradski, Adrian Kaehler. Learning OpenCV: Computer Vision with the OpenCV Library 
	\bibitem{opencv} OpenCV \url{http://opencv.willowgarage.com/wiki/}
	\bibitem{libdecodeqr} libdecodeqr \url{https://github.com/josephholsten/libdecodeqr}
	\bibitem{stadis} \url{http://es.scribd.com/doc/76878232/20/Table-1-Libdecodeqr-successful-statuses}
	\bibitem{pfcdaniel} ESTUDIO DE LOS CODIGOS QR \url{http://upcommons.upc.edu/pfc/bitstream/2099.1/14407/1/PFC\%20Daniel\%20Guti\%C3\%A9rrez\%20Garc\%C3\%ADa.pdf}
	\bibitem{libqr} \url{https://github.com/josephholsten/libdecodeqr}
	\bibitem{opencv1}\url{http://en.wikipedia.org/wiki/OpenCV}
	
	
	
	
	
	
	
	
	
	
	

	
	
	
	
	
	
	
	
	
	
	
	
	
\end{thebibliography}


\end{document}
